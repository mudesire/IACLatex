% This packagae is used to make list of symbols and list of abbreviations at the start of the report. 
% \makeglossaries
% This is the syntax for the symbol 
% \newglossaryentry{here comes the command you will use}
% {
%         name=here comes the symbol,
%         description={Is a mark up language specially suited for 
% scientific documents}
% }
% This is the syntax for abbreviations 
% \newacronym{here comes the command you will use}{here comes the abbereviation}{here comes the full form of the abbreviation}

% These are the commands to be used in the text:
% For the symbol  
% \gls{here comes the command you will use} -->  first letter small 
% \Gls{here comes the command you will use} -->  first letter caps
% \glspl{here comes the command you will use} -->  first letter small and plural 
% \Glspl{here comes the command you will use} -->  first letter caps and plural 

% For the abbreviations: 
% \acrshort{here comes the command you will use} -> will use the abbreviation 
% \acrlong{here comes the command you will use} -> will use the full form 
% \acrfull{here comes the command you will use} -> will use the full form (abbreviation).  

\newacronym{test}{TEST}{This section is not numbered. A nomenclature section could be provided when there are mathematical symbols in your paper. Superscripts and subscripts must be listed separately. Nomenclature definitions should not appear again in the text.
}
\newacronym{iac}{IAC}{This section is not numbered. A nomenclature section could be provided when there are mathematical symbols in your paper. Superscripts and subscripts must be listed separately. Nomenclature definitions should not appear again in the text.
}

\newglossaryentry{test_symbol}
{
        name=$\theta_{test}$, 
        description={This section is not numbered. Define acronyms and abbreviations that are not standard in this section. Such acronyms and abbreviations that are unavoidable in the abstract must be defined at their first mention there. Ensure consistency of abbreviations throughout the article. Always use the full title followed by the acronym (abbreviation) to be used, e.g., reusable suborbital launch vehicle (RSLV), International Space Station (ISS).}
}