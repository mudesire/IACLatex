\documentclass[]{iac}
% To make the list of abbreviations and symbols.
\usepackage[automake,acronym]{glossaries}
\makeglossaries
\include{abbreviations}
\makeglossaries
\DeclareMathOperator{\E}{E}
\DeclareMathOperator{\prob}{p}
\DeclareMathOperator{\tr}{tr}
\newcommand{\etalia}{\textit{et al.}}
\newcommand*{\vectornorm}[1]{\left\|#1\right\|}
\newcommand*\rfrac[2]{{{}^{#1}\!/_{#2}}} % running fraction with slash - requires math mode.
\newcommand*\T{\mathsf{T}}

\begin{document}

\IACpaperyear{20}
\IACpapernumber{B2.4.1}
\IACconference{71}
\IAClocation{The CyberSpace Edition, 12-14 October 2020}
\IACcopyrightA{2020}{International Astronautical Federation (IAF)}

\title{Manuscript Template and Style Guide (Title of your paper)}

\input{authors}

\abstract{~~~~Small satellite is part of a growing industry that caters to commercial, scientific and defense applications. Multiple launches and power constraints of small satellites require cost-effective and stable communication links.  Free-space laser communication (lasercom) is a promising candidate offering a high speed broadband network, with a less energy consuming and compact subsystem. Unlike radio frequency (RF), lasercom alleviates the licensing problems of RF spectrum regulation. Despite recognizable popularity, little research on lasercom for small satellites has been conducted on a comprehensive scale. This review discusses the current improvement in laser communication for small satellites. It examines innovative approaches in on-board Photonics for pointing strategies in atmospheric disturbances and vibrations mitigations. Due to the infeasibility of in-orbit maintenance, the crucial role of the multidisciplinary optimization of requirements for flexible and agile lasercom systems is addressed.}

%\IACkeywords{maximum 6 keywords}{}{}{}{}{}

\IACkeywords{laser technology}{inter-satellite links}{small satellite}{pointing}{design optimization}{laserCom}

\maketitle
% Add list of symbols
\printglossary[type=\acronymtype, title=Abbreviations]
\printglossary[title=Nomenclature]

\section{Introductions}
Section headings are in bold and placed flush on the left hand margin of the column.

The Introduction Section is to state the objectives of the work, provide an adequate background including a brief literature survey, major differences from the others, and sectional organization of this paper. Avoid a too detailed and lengthy literature survey and a summary of the results.

Divide your paper into clearly defined and numbered sections numbered 1., 2., …. Subsections should be numbered 1.1 (then 1.1.1, 1.1.2, ...), 1.2, etc. Use this numbering also for internal cross-referencing: do not just refer to “the text”. Any subsection may be given a brief heading. Each heading should appear on its own separate line.

\subsection{Subheadings}
Subheadings are underlined and placed flush on the left-hand margin of the column.

\subsubsection{Sub-subheadings}
Sub-subheadings are underlined and indented.

\section{Materials and methods}
Provide sufficient detail to allow the work to be reproduced. Methods already published should be indicated by a reference: only relevant modifications should be described.

\section{Theory and calculation}
A Theory section should extend, not repeat, the background to the article already dealt with in the Introduction and lay the foundation for further work. In contrast, a Calculation section represents a practical development from a theoretical basis.

\subsection{Equation Numbers}
When numbering equations, enclose numbers in brackets and place flush right with the right-hand margin of the column. The numbers identifying the equations should be placed in parentheses to the right of the equation. For example:

\begin{equation}
\stackrel{\star}{F}_{12} = -G \cdot \frac{ m_1 \cdot m_2 }{ \left\|\stackrel{\star}{r}_2 - \stackrel{\star}{r}_1\right\|^2 } \cdot \hat{u}_{12}
\end{equation}

\subsection{Illustrations and Captions}
It is important to remember that all artwork, captions, figures (\textit{e.g.}, Fig.~\ref{fig:X}), graphs, and tables (such as Table~\ref{table:X}) will be reproduced exactly as you submitted them. (\textbf{Company logos} and \textbf{identification numbers} are not permitted on your illustrations.)

\begin{figure}
    \includegraphics[width=\columnwidth]{examplefigure.png}
    \caption{\label{fig:X}Title of the figure, left-justified, subsequent text indented. Place figures at the top or bottom of a column wherever possible, as close as possible to the first references to them in the manuscript. Restrict them to single-column width unless this would make them illegible.}
\end{figure}

\subsection{Graph Lines, Drawings, and Tables}
Use black ink on white manuscript and position to fit within one of the columns on the page, and ensure that they remain still readable.

Tables with a moderate amount of information should be positioned within one column. Tables, graphs, or pictures with large amounts of information may extend across two columns.

\begin{table}
    \begin{tabular}{rllll}
        \toprule
        & Venus & Earth & Mars & Jupiter \\
        \midrule
        $\rfrac{M}{M_E}$	& 0.82 		& 1 		& 0.11 		& 317.89	\\
        $e$					& 0.007		& 0.017		& 0.093		& 0.048		\\
        $R$ (AU)			& 0.7233	& 1			& 1.524		& 5.203		\\
        $i$ (deg)			& 3.40		& 0			& 1.85		& 1.30		\\
        $T$ (years)			& 0.62		& 1			& 1.88		& 11.86		\\
        \bottomrule
    \end{tabular}
    \caption{\label{table:X}Title of table, left justified, subsequent text indented. Heading centered. Do not use vertical lines within the table; use horizontal lines only to separate headings from table entries.}
\end{table}

\section{Style Guide}

\subsection{Acronyms}
Always use the full title followed by the acronym to be used \acrfull{test}.

\subsection{Symbols}
Test the symbol \gls{test_symbol}

\subsection{References}
List and number all the bibliographical references at the end of the full text, in the order of appearance.\cite{Geim2001}

\subsection{Captions, Graph Axes, Legends}
Captions, graph axes, legends, \textit{etc.}, should be large enough to remain legible.

\subsection{Footnotes, Symbols, and Abbreviations}
Footnotes should be cited using symbols in this order: \footnote{Footnote 1} \footnote{Footnote 2} \footnote{Footnote 3} \footnote{Footnote 4} \footnote{Footnote 5} \footnote{Footnote 6} \footnote{Footnote 7} \footnote{Footnote 8}. Use only standard symbols and abbreviations in text and illustrations.

\subsection{Page Numbers}
Indicate page numbering at the bottom of each page.

\bibliographystyle{plain}
\bibliography{example}

\end{document}